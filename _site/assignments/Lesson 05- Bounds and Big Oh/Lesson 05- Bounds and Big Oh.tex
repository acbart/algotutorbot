\documentclass[12pt]{exam}

\newcommand{\Course}{CISC320}
\newcommand{\Semester}{Spring 2021}
\newcommand{\Assignment}{Lesson 05- Bounds and Big Oh}
\newcommand{\Version}{1.0.1}

\printanswers % If you want to print answers
 %\noprintanswers % If you don't want to print answers
\addpoints % if you want to count the points
% \noaddpoints % if you don't want to count the points
% Specifies the way question are displayed:

\qformat{\textbf{Question~\thequestion}\quad(\thepoints)\hfill}
\usepackage{color} % defines a new color
\definecolor{SolutionColor}{rgb}{0.9,.95,1} % light blue
%\shadedsolutions % defines the style of the solution environment
\framedsolutions % defines the style of the solution environment
% Defines the title of the solution environment:
\renewcommand{\solutiontitle}{\noindent\textbf{Solution:}\par\noindent}



\usepackage[letterpaper, margin=.5in]{geometry}
\usepackage[framemethod=tikz]{mdframed}
\usepackage{multicol}
\usepackage{amsmath}

\usepackage{listings}
\usepackage{color}
\definecolor{lightgray}{gray}{0.9}

\lstset{
    showstringspaces=false,
    basicstyle=\ttfamily,
    keywordstyle=\color{blue},
    commentstyle=\color[grey]{0.6},
    stringstyle=\color[RGB]{255,150,75}
}

\usepackage{tikz}
\usetikzlibrary{positioning}% To get more advances positioning options
\usetikzlibrary{arrows}% To get more arrow heads

\tikzstyle{circ} = [circle, rounded corners, draw=black]


\pagestyle{empty}

%\usepackage{pogilcs}

% Taken largely from CSPogil activities by Matt Lang
% Specifically Graph I activity.

\begin{document}

\makebox[.9\textwidth]{Names:\enspace\hrulefill}

\Course~-~\Semester~-~\Assignment~-~\Version


%%%%%%% BEGIN CONTENT


\begin{questions}


\question[2] True or False? Show your work!

\begin{parts}

\part $2^{n+1} = O(2^n)$
\begin{solution}[1.5in]
%%%% 1.a
\end{solution}


\part
$2^{2n} = O(2^n)$\\
\begin{solution}[1.5in]
%%%% 1.b
\end{solution}
 

\end{parts}

\question[3]  For each of the following pairs of functions, either $f(n)$ is in $O(g(n))$, f(n) is in $\Omega(g(n))$, or $f(n) = \Theta(g(n))$. Determine which relationship is correct and briefly explain why.

\begin{parts}

\part $f(n) = \sqrt{n} $ and $g(n) = \log{(n^2)}$

\begin{solution}[2in]
%%% 2.a
\end{solution}

\part $f(n) = 2^n $ and $g(n) = 3^n$

\begin{solution}[1in]
%%%% 2.b
\end{solution}

\part $f(n) = 2 \sqrt{(n)} + \log(n)$ and $g(n) = \sqrt{(n)} + 5$

\begin{solution}[3in]
%%%% 2.c
\end{solution}

\end{parts}

\question[2] Why is $n^2 = O(2^n)$

\begin{solution}[1in]

%%%%% 3

\end{solution}

\question[3] For each of the following pairs of functions $f(n)$ and $g(n)$, give a minimal positive integer constant $C$ such that $f(n) \leq C \cdot g(n)$ for all $n>1$.

\begin{parts}

\part $f(n)= n^2 + n + 1$ and $g(n) = 2n^3$

\begin{solution}[1in]

%%%%% 4.a

\end{solution}

\part $f(n) = n\cdot \sqrt{n} + n^2$ and $g(n) = n^2$

\begin{solution}[1in]

%%%% 4.b

\end{solution}

\part $f(n) = n^2 - n + 1$ and $g(n) = \frac{n^2}{2}$

\begin{solution}[1in]

%%%%% 4.c

\end{solution}

\end{parts}

\end{questions}

\end{document}
